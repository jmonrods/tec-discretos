\title[Circuitos Discretos]{Circuitos Discretos}
\subtitle{Presentación del Curso}
\institute[]{Instituto Tecnológico de Costa Rica\\Escuela de Ingeniería Electrónica\\Circuitos Discretos}
\date{Semestre I-2024}
\titlegraphic{\includegraphics[height=8mm]{logoTEC.png}}

\section{Presentación del curso}

\begin{frame}[t]
\titlepage
\end{frame}

\begin{frame}[t]
\frametitle{Información General}

\begin{table}
	\centering %\large
	\begin{tabular}{ll}
	\textbf{Curso} & Circuitos Discretos \\
	\textbf{Código} & EL-3212 \\
	\textbf{Tipo de curso} & Teórico \\
	\textbf{Créditos/horas por semana} & 4/4 \\
	\textbf{Requisitos} & EL-2207 Elementos Activos \\
	\textbf{Correquisitos} & Ninguno \\
	\textbf{Suficiencia} & Sí \\
	\textbf{Metodología} & Clases magistrales \\
	\textbf{Asistencia} & Obligatoria \\
	\textbf{Consulta} & Oficina 422 \\
	\textbf{Evaluación} & Exámenes escritos, individual \\
	\end{tabular}
\end{table}
\end{frame}

\begin{frame}[t]
\frametitle{Objetivos}
\textbf{Descripción del curso}

Se estudian los conceptos de amplificación electrónica y procesamiento de señales utilizando transistores BJT y MOSFET. Se definen los parámetros típicos que caracterizan a un amplificador en distintas condiciones de trabajo, y se ofrecen las herramientas básicas de dimensionamiento para construir amplificadores de pequeña y gran señal de pequeña y mediana complejidad. Además, se estudian los conceptos de respuesta en frecuencia, realimentación y estabilidad y la aplicación de los mismos en el desarrollo de circuitos analógicos avanzados.

\textbf{Objetivo general}

Al terminar este curso, el estudiante debe ser capaz de definir y evaluar un amplificador electrónico y sus parámetros característicos, utilizando los modelos compactos de primer y segundo orden de los transistores BJT y MOSFET, apoyándose en métodos de análisis de circuitos para aplicar dichos parámetros en un rango determinado de operación. Esta capacidad deberá a su vez generar criterios básicos de diseño y evaluación de amplificadores electrónicos para aplicaciones avanzadas en sistemas complejos de control, comunicaciones y adquisición y procesamiento de señales.
\end{frame}


\begin{frame}[t]
\frametitle{Contenidos}
\textbf{Definiciones generales de los amplificadores electrónicos y de los parámetros de caracterización de un amplificador (1 semana)}

\vspace{5mm}
\textbf{Amplificadores electrónicos básicos con transistores de silicio (BJT y MOSFET) (4 semanas)}

\begin{itemize}
	\item Introducción a los amplificadores de señal.
	\item Repaso de polarización.
	\item Modelos compactos para pequeña y gran señal.
	\item Estudio de configuraciones básicas de amplificadores BJT: emisor, colector y base común.
	\item Configuración básica de amplificadores MOSFET: fuente, drenaje y compuerta común
\end{itemize}

\end{frame}

\begin{frame}[t]
\frametitle{Contenidos}

\textbf{Amplificadores avanzados y otras estructuras (3 semanas)}

\begin{itemize}
	\item Espejos de corriente.
	\item Amplificador cascodo.
	\item Amplificador diferencial.
\end{itemize}

\textbf{Respuesta de frecuencia (3 semanas)}

\begin{itemize}
	\item Conceptos fundamentales de respuesta de frecuencia y teorema de Miller.
	\item Modelos de alta frecuencia de los transistores BJT y MOSFET y frecuencia de tránsito.
	\item Respuesta de frecuencia de configuraciones básicas.
	\item Respuesta de etapas cascodo y pares diferenciales.	
\end{itemize}
\end{frame}

\begin{frame}[t]
\frametitle{Contenidos}

\textbf{Realimentación negativa (3 semanas)}

\begin{itemize}
	\item Consideraciones generales y propiedades básicas de la realimentación negativa.
	\item Efectos de la realimentación sobre las características básicas de un amplificador.
	\item Técnicas de sensado y retorno.
	\item Topologías de realimentación y problemas de estabilidad		
\end{itemize}

\textbf{Etapas de salida y amplificadores de potencia (1 semanas)}

\begin{itemize}
	\item Etapa push-pull.
	\item Consideraciones de gran señal.
	\item Disipación de calor y eficiencia		
\end{itemize}

\textbf{Circuitos de realimentación positiva (1 semana)}

\begin{itemize}
	\item Multivibradores.
	\item Osciladores.	
\end{itemize}

\end{frame}

\begin{frame}[t]
\frametitle{Evaluación y Bibliografía}
\textbf{Evaluación} 

\begin{itemize}
	\item Examen parcial I: 30\%
	\item Examen parcial II: 30\%
	\item Examen parcial III: 30\%
	\item Tareas: 10\%.		
\end{itemize}

\vspace{5mm}
\textbf{Bibliografía obligatoria}

[1] Behzad R. Fundamentals of Microelectronics, 2da ed. Wiley, 2013.

\vspace{5mm}
\textbf{Bibliografía complementaria}

[2] Behzad Razavi. Design of Analog CMOS Integrated Circuits, 2da ed. McGraw Hill Education, 2016.

[3] Adel S. Sedra. Circuitos microelectrónicos. 7a ed. Oxford University Press, 2014.

\end{frame}

