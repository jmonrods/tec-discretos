\section{Clase 5}

\title[Circuitos Discretos]{Circuitos Discretos}
\subtitle{Clase 5: Emisor Común con Polarización}
\institute[]{Instituto Tecnológico de Costa Rica\\Escuela de Ingeniería Electrónica\\Circuitos Discretos}
\date{\theSemester}
\titlegraphic{\includegraphics[height=8mm]{logoTEC.png}}

\begin{frame}[t]
    \titlepage
\end{frame}

\begin{frame}[t]
    \frametitle{Amplificador de Emisor Común}

    Si se conecta un micrófono y un parlante, ¿el circuito funciona?

    \vspace{5mm}
    \centering
    \begin{figure}[H]
        \begin{circuitikz}
            \draw (0,0) node[npn](npn1){};
            \draw (-2,0) to[short] (npn1.base);
            \draw (-2,-2) to[mic,l=$mic$] (-2,0);
            \draw (-2,-2) node[ground]{};
            \draw (0,2) to[R,l=$R_C$] (npn1.collector);
            \draw (0,0.5) to[short] (2,0.5);
            \draw (2,0.5) to[loudspeaker,l=$8\ \Omega$] (2,-1.5);
            \draw (2,-1.5) node[ground]{};
            \draw (0,-1) node[ground]{};
            \draw (0,-1) -- (npn1.emitter);
            \draw (0,2.5) -- (0,2);
            \draw (-0.5,2.5) -- (0.5,2.5);
            \draw (0,2.5) node[anchor=south]{$V_{CC}$};
        \end{circuitikz}
    \end{figure}
\end{frame}


\begin{frame}[t]
    \frametitle{Amplificador de Emisor Común}

    Si se conecta un micrófono y un parlante, ¿el circuito funciona?

    \vspace{5mm}
    \begin{columns}
        \begin{column}{0.5\textwidth}
            \begin{figure}[H]
                \centering
                \begin{circuitikz}
                    \draw (0,0) node[npn](npn1){};
                    \draw (-1.5,0) to[short] (npn1.base);
                    \draw (-1.5,-2) to[mic,l=$mic$] (-1.5,0);
                    \draw (-1.5,-2) node[ground]{};
                    \draw (0,2) to[R,l=$R_C$] (npn1.collector);
                    \draw (0,0.5) to[short] (1.5,0.5);
                    \draw (1.5,0.5) to[loudspeaker,l=$8\ \Omega$] (1.5,-1.5);
                    \draw (1.5,-1.5) node[ground]{};
                    \draw (0,-1) node[ground]{};
                    \draw (0,-1) -- (npn1.emitter);
                    \draw (0,2.5) -- (0,2);
                    \draw (-0.5,2.5) -- (0.5,2.5);
                    \draw (0,2.5) node[anchor=south]{$V_{CC}$};
                \end{circuitikz}
            \end{figure}
        \end{column}
        \begin{column}{0.5\textwidth}
            Sin un circuito de polarización, no existe una corriente $I_C$.

            \begin{itemize}
                \item El punto de operación en gran señal está en el origen.
                \item La transconductancia es cero.
                \item No hay amplificación.
            \end{itemize}

            \vspace{3mm}
            Aun si se agrega un circuito de polarización, la baja impedancia del parlante hace que la ganancia disminuya de manera considerable.
            %
            \[ A_V = - g_m (R_C \parallel 8\ \Omega) \approx -g_m \cdot 8\ \Omega \]
        \end{column}
    \end{columns}
    
\end{frame}

\begin{frame}[t]
    \frametitle{Condensadores de Desacople}
    \begin{columns}
        \begin{column}{0.5\textwidth}
            Con el parlante conectado, también existe un problema de polarización:

            \begin{itemize}
                \item El parlante está formado por una inductancia, que en CD es casi un corto circuito en la salida. 
                \item El micrófono es una fuente de CA, que en gran señal se convertiría en un corto circuito en la entrada.
            \end{itemize}
            
            \vspace{3mm}
            Para evitar este problema se utilizan \textbf{condensadores de desacople}. 
            
            \begin{itemize}
                \item Un condensador en CC (gran señal) es un circuito abierto.
                \item Un condensador en CA (pequeña señal) es un cortocircuito. 
            \end{itemize}
        \end{column}
        \begin{column}{0.5\textwidth}
            \begin{figure}[H]
                \scalebox{0.85}{
                \begin{circuitikz}
                    \draw (0,0) node[npn](npn1){};
                    \draw (-2,0) to[C] (npn1.base);
                    \draw (-2,-2) to[mic,l=$mic$] (-2,0);
                    \draw (-2,-2) node[ground]{};
                    \draw (0,2) to[R,l=$R_C$] (npn1.collector);
                    \draw (0,0.5) to[C] (2,0.5);
                    \draw (2,0.5) to[loudspeaker,l=$8\ \Omega$] (2,-1.5);
                    \draw (2,-1.5) node[ground]{};
                    \draw (0,-1) node[ground]{};
                    \draw (0,-1) -- (npn1.emitter);
                    \draw (0,2.5) -- (0,2);
                    \draw (-0.5,2.5) -- (0.5,2.5);
                    \draw (0,2.5) node[anchor=south]{$V_{CC}$};
                \end{circuitikz}
                }
            \end{figure}
        \end{column}
    \end{columns}
    
\end{frame}

\begin{frame}[t]
    \frametitle{Pasos de Solución de Circuitos Amplificadores}

    \textbf{Pasos de solución}

    \vspace{5mm}
    1. Se analiza el circuito en gran señal.

    \begin{itemize}
        \item Condensadores son un circuito abierto.
        \item Fuentes de AC apagadas.
        \item Se calcula el punto de operación ($I_C$ y $V_{BE}$).
    \end{itemize}

    \vspace{5mm}
    2. Se calculan los parámetros de pequeña señal: $g_m$, $r_\pi$, $r_o$.

    \vspace{5mm}
    3. Se analiza el circuito en pequeña señal.

    \begin{itemize}
        \item Condensadores son un corto circuito.
        \item Fuentes de DC apagadas.
        \item Se calcula la ganancia y las impedancias de entrada y salida.
    \end{itemize}
\end{frame}


\begin{frame}[t]
    \frametitle{Emisor Común: Polarización por Resistencia de Base}

    \centering
    \begin{figure}[H]
        \begin{circuitikz}
            \draw (0,0) node[npn](npn1){};
            \draw (-4,0) to[C] (-2,0);
            \draw (-2,0) -- (npn1.base);
            \draw (-2,2) to[R,l=$R_B$] (-2,0);
            \draw (-2,2) -- (-2,2.5);
            \draw (-4,-2) to[mic,l=$mic$] (-4,0);
            \draw (-4,-2) node[ground]{};
            \draw (0,2) to[R,l=$R_C$] (npn1.collector);
            \draw (0,0.5) to[C] (2,0.5);
            \draw (2,0.5) to[loudspeaker,l=$R_L$] (2,-1.5);
            \draw (2,-1.5) node[ground]{};
            \draw (0,-1) node[ground]{};
            \draw (0,-1) -- (npn1.emitter);
            \draw (0,2.5) -- (0,2);
            \draw (-2.5,2.5) -- (0.5,2.5);
            \draw (-1,2.5) node[anchor=south]{$V_{CC}$};
        \end{circuitikz}
    \end{figure}

    \flushleft
    Para el circuito mostrado anteriormente, calcule:

    \begin{itemize}
        \item El punto de operación ($I_C$, $V_{BE}$)
        \item Los parámetros de pequeña señal ($g_m$, $r_\pi$, $r_o$)
        \item La ganancia $A_V$, impedancia de entrada $R_{in}$ e impedancia de salida $R_{out}$
    \end{itemize}

\end{frame}

\begin{frame}[t]
    \frametitle{Emisor Común: Polarización por Resistencia de Base}

    \begin{columns}
        \begin{column}{0.5\textwidth}
            \centering
            \textbf{Gran Señal}

            \begin{figure}[H]
                \scalebox{0.8}{
                \begin{circuitikz}
                    \draw (0,0) node[npn](npn1){};
                    \draw (-2,0) -- (npn1.base);
                    \draw (-2,2) to[R,l=$R_B$] (-2,0);
                    \draw (-2,2) -- (-2,2.5);
                    \draw (0,2) to[R,l=$R_C$] (npn1.collector);
                    \draw (0,-1) node[ground]{};
                    \draw (0,-1) -- (npn1.emitter);
                    \draw (0,2.5) -- (0,2);
                    \draw (-2.5,2.5) -- (0.5,2.5);
                    \draw (-1,2.5) node[anchor=south]{$V_{CC}$};
                \end{circuitikz}
                }
            \end{figure}
            %
            \[ \left\{ \begin{matrix}
                I_C = \beta \left( \dfrac{V_{CC}-V_{BE}}{R_B} \right) \\
                V_{BE} = V_t \ln (I_C / I_S)
            \end{matrix} \right. \]
            %
            \[ \begin{matrix}
                g_m = I_C / V_t \\
                r_\pi = \beta / g_m \\
                r_o = V_A / I_C
            \end{matrix} \]
            \vspace{3mm}
        \end{column}
        \begin{column}{0.5\textwidth}
            \centering
            \textbf{Pequeña Señal}

            \begin{figure}[H]
                \scalebox{0.8}{
                \begin{circuitikz}
                    \draw (0,0) node[npn](npn1){};
                    \draw (-3,0) -- (-2,0);
                    \draw (-2,0) -- (npn1.base);
                    \draw (-2,0) to[R,l=$R_B$] (-2,-2);
                    \draw (-2,-2) node[ground]{};
                    \draw (-3,-2) to[mic,l=$mic$] (-3,0);
                    \draw (-3,0) node[anchor=south]{$v_{in}$};
                    \draw (-3,-2) node[ground]{};
                    \draw (0,2) to[R,l=$R_C$] (npn1.collector);
                    \draw (0,0.5) to[short,-o] (2,0.5);
                    \draw (2,0.5) node[anchor=south]{$v_{out}$};
                    \draw (1,0.5) to[loudspeaker,l=$R_L$] (1,-1.5);
                    \draw (1,-1.5) node[ground]{};
                    \draw (0,-1) node[ground]{};
                    \draw (0,-1) -- (npn1.emitter);
                    \draw (0,2.5) -- (0,2);
                    \draw (0,2.5) -- (-1,2.5);
                    \draw (-1,2.5) node[ground]{};
                \end{circuitikz}
                }
            \end{figure}
            %
            \[ A_V = -g_m (R_C \parallel r_o \parallel R_L) \]
            %
            \[ R_{in} = R_B \parallel r_\pi \]
            %
            \[ R_{out} = R_C \parallel r_o \]
        \end{column}
    \end{columns}
\end{frame}

\begin{frame}[t]
    \frametitle{Emisor Común: Impedancias de Entrada y Salida}

    \centering
    \begin{figure}[H]
        \begin{circuitikz}
            \draw (0,0) node[npn](npn1){};
            \draw (-4,0) to[C] (-2,0);
            \draw (-2,0) -- (npn1.base);
            \draw (-2,2) to[R,l=$R_B$] (-2,0);
            \draw (-2,2) -- (-2,2.5);
            \draw (-5,0) -- (-4,0);
            \draw (-5,-2) to[mic,l=$mic$] (-5,0);
            \draw (-5,-2) node[ground]{};
            \draw (0,2) to[R,l=$R_C$] (npn1.collector);
            \draw (0,0.5) to[C] (2,0.5);
            \draw (2,0.5) -- (3,0.5);
            \draw (3,0.5) to[loudspeaker,l=$R_L$] (3,-1.5);
            \draw (3,-1.5) node[ground]{};
            \draw (0,-1) node[ground]{};
            \draw (0,-1) -- (npn1.emitter);
            \draw (0,2.5) -- (0,2);
            \draw (-2.5,2.5) -- (0.5,2.5);
            \draw (-1,2.5) node[anchor=south]{$V_{CC}$};
            % Recuadro del amplificador
            \draw[dashed] (-4,-2) rectangle (2,3.5);
            % Impedancia de entrada
            \draw[->] (-4.7,0.5) -- (-4.2,0.5);
            \draw (-4.7,0.5) -- (-4.7,1.5);
            \draw (-4.7,1.5) node[anchor=east]{$R_{in}$};
            % Impedancia de salida
            \draw[->] (2.8,1) -- (2.3,1);
            \draw (2.8,1) -- (2.8,2);
            \draw (2.8,2) node[anchor=west]{$R_{out}$};
        \end{circuitikz}
    \end{figure}

    \flushleft
    Las impedancias de entrada y salida son las que se observan hacia adentro de las terminales del amplificador, \textbf{omitiendo la fuente y la carga.}
\end{frame}

\begin{frame}[t]
    \frametitle{Emisor Común: Degeneración de Emisor y $R_S$}
    
    Considere que se agrega degeneración de emisor ($R_E$) y que la fuente tiene una impedancia no ideal:

    \centering
    \begin{figure}[H]
        \begin{circuitikz}
            \draw (0,0) node[npn](npn1){};
            \draw (-4,0) to[C] (-2,0);
            \draw (-2,0) -- (npn1.base);
            \draw (-2,2) to[R,l=$R_B$] (-2,0);
            \draw (-2,2) -- (-2,2.5);
            \draw (-6,-2) to[mic,l=$mic$] (-6,0);
            \draw (-5.3,0) to[R,l=$R_S$] (-4,0);
            \draw (-5.3,0) -- (-6,0);
            \draw (-6,-2) node[ground]{};
            \draw (0,2) to[R,l=$R_C$] (npn1.collector);
            \draw (0,0.5) to[C] (2,0.5);
            \draw (2,0.5) to[loudspeaker,l=$R_L$] (2,-1.5);
            \draw (2,-1.5) node[ground]{};
            \draw (0,-2) node[ground]{};
            \draw (npn1.emitter) to[R,l=$R_E$] (0,-2);
            \draw (0,2.5) -- (0,2);
            \draw (-2.5,2.5) -- (0.5,2.5);
            \draw (-1,2.5) node[anchor=south]{$V_{CC}$};
        \end{circuitikz}
    \end{figure}
\end{frame}

\begin{frame}[t]
    \frametitle{Emisor Común: Degeneración de Emisor y $R_S$}
    
    En pequeña señal, el circuito queda como sigue:

    \centering
    \vspace{3mm}
    \begin{figure}[H]
        \begin{circuitikz}
            \draw (0,0) node[npn](npn1){};
            \draw (-3,0) -- (npn1.base);
            \draw (-3,0) to[R,l=$R_B$] (-3,-2);
            \draw (-3,-2) node[ground]{};
            \draw (-6,-2) to[mic,l=$mic$] (-6,0);
            \draw (-6,0) to[R,l=$R_S$] (-3,0);
            \draw (-6,-2) node[ground]{};
            \draw (0,2) to[R,l=$R_C$] (npn1.collector);
            \draw (0,0.5) -- (2,0.5);
            \draw (2,0.5) to[loudspeaker,l=$R_L$] (2,-1.5);
            \draw (2,-1.5) node[ground]{};
            \draw (0,-2) node[ground]{};
            \draw (npn1.emitter) to[R,l=$R_E$] (0,-2);
            \draw (0,2.5) -- (0,2);
            \draw (-1,2.5) -- (0,2.5);
            \draw (-1,2.5) node[ground]{};
            \draw (-4.5,1.5) node[anchor=south]{ $ R_{in} = R_B \parallel \left[r_\pi + (\beta+1)R_E\right]$ };
            \draw (-4.5,1) node[anchor=south]{ $ R_{out} = R_C \parallel r_o \left[1 + g_m (R_E \parallel r_\pi)\right] $ };
        \end{circuitikz}
    \end{figure}
    
    \[ A_V = \dfrac{-R_C \parallel R_L \parallel r_o(1+g_m (R_E\parallel r_\pi))}{1/g_m + R_E} \times \dfrac{R_B \parallel \left[r_\pi + (\beta+1)R_E\right]}{R_S + R_B \parallel \left[r_\pi + (\beta+1)R_E\right]} \]
\end{frame}


\begin{frame}[t]
    \frametitle{Emisor Común: Polarización por Divisor de Tensión}

    \begin{figure}[H]
        \scalebox{1}{
        \begin{circuitikz}
            \draw (0,0) node[npn](npn1){};
            \draw (-2,0) -- (npn1.base);
            \draw (-2,0) to[R,l=$R_2$] (-2,-2);
            \draw (-2,-2) node[ground]{};
            \draw (-4,0) to[C] (-2,0);
            \draw (-5.3,0) to[R,l=$R_S$] (-4,0);
            \draw (-5.3,0) -- (-6,0);
            \draw (-6,-2) to[mic,l=$mic$] (-6,0);
            \draw (-6,0) node[anchor=south]{$v_{in}$};
            \draw (-6,-2) node[ground]{};
            \draw (0,2) to[R,l=$R_C$] (npn1.collector);
            \draw (0,0.5) to[C] (2,0.5);
            \draw (2,0.5) to[short,-o] (3,0.5);
            \draw (3,0.5) node[anchor=south]{$v_{out}$};
            \draw (2,0.5) to[loudspeaker,l=$R_L$] (2,-1.5);
            \draw (2,-1.5) node[ground]{};
            \draw (0,-2) node[ground]{};
            \draw (npn1.emitter) to[R,l=$R_E$] (0,-2);
            \draw (0,2.5) -- (0,2);
            \draw (-2.5,2.5) -- (0.5,2.5);
            \draw (-1,2.5) node[anchor=south]{$V_{CC}$};
            \draw (-2,2.5) -- (-2,2);
            \draw (-2,2) to[R,l=$R_1$] (-2,0);
        \end{circuitikz}
        }
    \end{figure}
\end{frame}


\begin{frame}[t]
    \frametitle{Emisor Común: Polarización por Divisor de Tensión}

    \begin{figure}[H]
        \scalebox{1}{
        \begin{circuitikz}
            \draw (0,0) node[npn](npn1){};
            \draw (-2,0) -- (npn1.base);
            \draw (-3,0) to[R,l=$R_1 \parallel R_2$] (-3,-2);
            \draw (-3,-2) node[ground]{};
            \draw (-4,0) to[short] (-2,0);
            \draw (-5.3,0) to[R,l=$R_S$] (-4,0);
            \draw (-5.3,0) -- (-6,0);
            \draw (-6,-2) to[mic,l=$mic$] (-6,0);
            \draw (-6,0) node[anchor=south]{$v_{in}$};
            \draw (-6,-2) node[ground]{};
            \draw (0,2) to[R,l=$R_C$] (npn1.collector);
            \draw (0,0.5) to[short] (2,0.5);
            \draw (2,0.5) to[short,-o] (3,0.5);
            \draw (3,0.5) node[anchor=south]{$v_{out}$};
            \draw (2,0.5) to[loudspeaker,l=$R_L$] (2,-1.5);
            \draw (2,-1.5) node[ground]{};
            \draw (0,-2) node[ground]{};
            \draw (npn1.emitter) to[R,l=$R_E$] (0,-2);
            \draw (0,2.5) -- (0,2);
            % Fuente de alimentación en AC
            \draw (-1.5,2.5) -- (0,2.5);
            \draw (-1.5,2.5) node[ground]{};
            \draw (-5.5,1.5) node[anchor=south]{ $ R_{in} = R_1 \parallel R_2 \parallel \left[r_\pi + (\beta+1)R_E\right]$ };
            \draw (-5.5,1) node[anchor=south]{ $ R_{out} = R_C \parallel r_o \left[1 + g_m (R_E \parallel r_\pi)\right] $ };
        \end{circuitikz}
        }
    \end{figure}

    Solución 1: divisor de tensión en la entrada
    %
    \[ A_V = \dfrac{-R_C \parallel R_L \parallel r_o(1+g_m (R_E\parallel r_\pi))}{1/g_m + R_E} \times \dfrac{(R_1 \parallel R_2) \parallel \left[r_\pi + (\beta+1)R_E\right]}{R_S + (R_1 \parallel R_2) \parallel \left[r_\pi + (\beta+1)R_E\right]} \]
\end{frame}

\begin{frame}[t]
    \frametitle{Emisor Común: Polarización por Divisor de Tensión}

    \begin{figure}[H]
        \scalebox{1}{
        \begin{circuitikz}
            \draw (0,0) node[npn](npn1){};
            \draw (-2,0) -- (npn1.base);
            \draw (-3,0) to[R,l=$R_1 \parallel R_2$] (-3,-2);
            \draw (-3,-2) node[ground]{};
            \draw (-4,0) to[short] (-2,0);
            \draw (-5.3,0) to[R,l=$R_S$] (-4,0);
            \draw (-5.3,0) -- (-6,0);
            \draw (-6,-2) to[mic,l=$mic$] (-6,0);
            \draw (-6,0) node[anchor=south]{$v_{in}$};
            \draw (-6,-2) node[ground]{};
            \draw (0,2) to[R,l=$R_C$] (npn1.collector);
            \draw (0,0.5) to[short] (2,0.5);
            \draw (2,0.5) to[short,-o] (3,0.5);
            \draw (3,0.5) node[anchor=south]{$v_{out}$};
            \draw (2,0.5) to[loudspeaker,l=$R_L$] (2,-1.5);
            \draw (2,-1.5) node[ground]{};
            \draw (0,-2) node[ground]{};
            \draw (npn1.emitter) to[R,l=$R_E$] (0,-2);
            \draw (0,2.5) -- (0,2);
            % Fuente de alimentación en AC
            \draw (-1.5,2.5) -- (0,2.5);
            \draw (-1.5,2.5) node[ground]{};
            \draw (-5.5,1.5) node[anchor=south]{ $ R_{in} = R_1 \parallel R_2 \parallel \left[r_\pi + (\beta+1)R_E\right]$ };
            \draw (-5.5,1) node[anchor=south]{ $ R_{out} = R_C \parallel r_o \left[1 + g_m (R_E \parallel r_\pi)\right] $ };
        \end{circuitikz}
        }
    \end{figure}

    Solución 2: resistencias de base aparecen en serie con el emisor (entre $\beta+1$)
    %
    \[ A_V = \dfrac{-R_C \parallel R_L \parallel r_o (1+g_m(R_E\parallel r_\pi))}{\dfrac{1}{g_m} + R_E + \dfrac{R_1 \parallel R_2 \parallel R_S}{\beta + 1}} \]
\end{frame}



\begin{frame}[t]
    \frametitle{Emisor Común: Condensador de Derivación (Bypass)}

    \begin{figure}[H]
        \scalebox{0.95}{
        \begin{circuitikz}
            \draw (0,0) node[npn](npn1){};
            \draw (-2,0) -- (npn1.base);
            \draw (-2,2) to[R,l=$R_1$] (-2,0);
            \draw (-2,0) to[R,l=$R_2$] (-2,-2);
            \draw (-2,-2) node[ground]{};
            \draw (-4,0) to[C] (-2,0);
            \draw (-5.3,0) to[R,l=$R_S$] (-4,0);
            \draw (-5.3,0) -- (-6,0);
            \draw (-6,-2) to[mic,l=$mic$] (-6,0);
            \draw (-6,0) node[anchor=south]{$v_{in}$};
            \draw (-6,-2) node[ground]{};
            \draw (0,2) to[R,l=$R_C$] (npn1.collector);
            \draw (0,0.5) to[C] (2,0.5);
            \draw (2,0.5) to[short,-o] (4,0.5);
            \draw (4,0.5) node[anchor=south]{$v_{out}$};
            \draw (3,0.5) to[loudspeaker,l=$R_L$] (3,-1.5);
            \draw (3,-1.5) node[ground]{};
            \draw (0,-2) node[ground]{};
            \draw (npn1.emitter) to[R,l=$R_E$] (0,-2);
            % Condensador de derivacion
            \draw (0,-0.5) -- (1.5,-0.5);
            \draw (1.5,-0.5) to[C,l=$C_b$] (1.5,-2);
            \draw (1.5,-2) node[ground]{};
            % Fuente de alimentacion
            \draw (0,2.5) -- (0,2);
            \draw (-2.5,2.5) -- (0.5,2.5);
            \draw (-1,2.5) node[anchor=south]{$V_{CC}$};
            \draw (-2,2.5) -- (-2,2);
            % Ecuaciones Rin y Rout
            \draw (-5.5,1.5) node[anchor=south]{ $ R_{in} = R_1 \parallel R_2 \parallel r_\pi$ };
            \draw (-5.5,1) node[anchor=south]{ $ R_{out} = R_C \parallel r_o $ };
        \end{circuitikz}
        }
    \end{figure}

    \[ A_V = -g_m (R_C \parallel R_L \parallel r_o) \times \dfrac{(R_1 \parallel R_2) \parallel r_\pi}{R_S + (R_1 \parallel R_2) \parallel r_\pi} \]
\end{frame}

\begin{frame}[t]
    \frametitle{Emisor Común: Ejemplo}

    Para el circuito mostrado en la figura, determine el punto de operación y calcule los siguientes parámetros: $g_m$, $r_\pi$, $r_o$, $v_{out}/v_B$, $v_B/v_{in}$ y $A_V$.

    \begin{figure}[H]
        \begin{circuitikz}
            \draw (0,0) node[npn](npn1){};
            \draw (-4,0) to[C] (-2,0);
            \draw (-6,0) to[R,l=$R_S$,a=$1\ k\Omega$] (-4,0);
            \draw (-6,0) to[short,-o] (-7,0);
            \draw (-7,0) node[anchor=east]{$v_{in}$};
            \draw (-2,0) -- (npn1.base);
            \draw (-2,2) to[R,l=$R_B$,a=$50\ k\Omega$] (-2,0);
            \draw (-2,2) -- (-2,2.5);
            \draw (0,2) to[R,l=$R_C$,a=$1\ k\Omega$] (npn1.collector);
            \draw (0,-0.5) -- (2,-0.5);
            \draw (2,-0.5) to[C,l=$C_b$] (2,-2);
            \draw (2,-2) node[ground]{};
            \draw (0,0.5) to[short,-o] (3,0.5);
            \draw (3,0.5) node[anchor=west]{$v_{out}$};
            \draw (npn1.emitter) to[R,l=$R_E$,a=$2\ k\Omega$] (0,-2) ;
            \draw (0,-2) node[ground]{};
            \draw (0,2.5) -- (0,2);
            \draw (-2.5,2.5) -- (0.5,2.5);
            \draw (-1,2.5) node[anchor=south]{$V_{CC}=2.5\ V$};
        \end{circuitikz}
    \end{figure}
    
    Considere que el transistor tiene $I_S=8\times{}10^{-16}\ A$, $\beta=100$, $V_A=5\ V$.
\end{frame}

\begin{frame}[t]
    \frametitle{Emisor Común: Solución}

    \begin{columns}
        \begin{column}{0.5\textwidth}
            \textbf{Punto de operación}
            %
            \[ I_B = \beta \dfrac{V_{CC}-V_{BE}-I_E R_E}{R_B} \]
            %
            \[ I_C = 708\ \mu A \]
            %
            \[ V_{BE} = 715\ mV \]
            %
            \vspace{3mm}
            \textbf{Parámetros $\pi$}
            %
            \[ g_m = \dfrac{I_C}{V_t} = \dfrac{708\ \mu A}{26\ mV} = \boxed{27.23\ mS} \]
            %
            \[ r_\pi = \dfrac{\beta}{g_m} = \dfrac{100}{27.23\ mS} = \boxed{3.67\ k\Omega} \]
            %
            \[ r_o = \dfrac{V_A}{I_C} = \dfrac{5\ V}{708\ \mu A} = \boxed{7.06\ k\Omega} \]
        \end{column}
        \begin{column}{0.5\textwidth}
            \textbf{Ganancia}
            %
            \[ A_V = \dfrac{v_{out}}{v_B}\times{}\dfrac{v_B}{v_{in}} \]
            %
            \[ A_V = -g_m (R_C \parallel r_o) \times{}\dfrac{R_B\parallel r_\pi}{R_S+R_B\parallel r_\pi} \]
            %
            \[ \dfrac{v_{out}}{v_B} = -(27.23\ mS) (1\ k\Omega \parallel 7.06\ k\Omega) \]
            %
            \[ \dfrac{v_{out}}{v_B} = -23.85 \]
            %
            \[ \dfrac{v_B}{v_{in}} = \dfrac{50\ k\Omega \parallel 3.67\ k\Omega}{1\ k\Omega+50\ k\Omega \parallel 3.67\ k\Omega} \]
            %
            \[ \dfrac{v_B}{v_{in}} = 0.77 \]
            %
            \[ \boxed{A_V = 18.36} \]
        \end{column}
    \end{columns}
\end{frame}

%\begin{frame}[t]
%    \frametitle{Etapas E.C. en cascada}
%\end{frame}
