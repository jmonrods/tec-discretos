\section{Clase 8}

\title[Circuitos Discretos]{Circuitos Discretos}
\subtitle{Clase 8: Fuente Común}
\institute[]{Instituto Tecnológico de Costa Rica\\Escuela de Ingeniería Electrónica\\Circuitos Discretos}
\date{\theSemester}
\titlegraphic{\includegraphics[height=8mm]{logoTEC.png}}

\begin{frame}[t]
\titlepage
\end{frame}


\begin{frame}[t]
    \frametitle{Fuente Común}

    \begin{columns}
        \begin{column}{0.35\textwidth}
            \centering
            \begin{figure}[H]
                \begin{circuitikz}
                    \draw (0,0) node[nmos](nmos1){};
                    \draw (-1.5,0) to[short] (nmos1.gate);
                    \draw (-1.5,0) to[vsource,l=$v_{in}$] (-1.5,-2);
                    \draw (-1.5,-2) node[ground]{};
                    \draw (0,2) to[R,l=$R_D$] (nmos1.drain);
                    \draw (0,0.5) to[short,-o] (1,0.5);
                    \draw (0,-1) node[ground]{};
                    \draw (0,-1) -- (nmos1.source);
                    \draw (1,0.5) node[anchor=west]{$v_{out}$};
                    \draw (0,2.5) -- (0,2);
                    \draw (-0.5,2.5) -- (0.5,2.5);
                    \draw (0,2.5) node[anchor=south]{$V_{DD}$};
                \end{circuitikz}
            \end{figure}
        \end{column}
        \begin{column}{0.65\textwidth}
            Para la configuración de Fuente Común:

            \begin{itemize}
                \item La entrada se aplica en la compuerta.
                \item La salida se toma del drenador.
            \end{itemize}

            \vspace{5mm}

            Análisis cualitativo de $A_V$:

            \begin{enumerate}
                \item Si $v_{in}$ aumenta, $v_{GS}$ aumenta
                \item Si $v_{GS}$ aumenta, $i_D$ aumenta
                \item Si $i_D$ aumenta, $v_{RD}$ aumenta
                \item Si $v_{RD}$ aumenta, $v_{out}$ disminuye
            \end{enumerate}

            $\Rightarrow$ Un cambio positivo en $v_{in}$ reduce $v_{out}$

            $\Rightarrow$ La ganancia es negativa
        \end{column}
    \end{columns}
\end{frame}

\begin{frame}[t]
    \frametitle{Fuente Común: Ganancia del Núcleo}

    \begin{columns}
        \begin{column}{0.35\textwidth}
            1. LVK por la compuerta
    
            \[ v_{in} = v_{GS} \]

            2. LCK en el colector

            \[ g_m v_{GS} + \dfrac{v_{out}}{R_D} = 0 \]

            3. Combinando 1 y 2:
    
            \[ g_m v_{in} + \dfrac{v_{out}}{R_D} = 0 \]

            \[ \dfrac{v_{out}}{R_D} = -g_m v_{in} \]

            \[ \boxed{A_V = \dfrac{v_{out}}{v_{in}} = -g_m R_D} \]
        \end{column}
        \begin{column}{0.65\textwidth}
            \centering
            \begin{figure}[H]
                \begin{circuitikz}
                    \draw (0,0) -- (1,0);
                    \draw (1,0) to[open,v=$v_{gs}$] (1,-2);
                    \draw (1,-2) -- (3,-2);
                    \draw (3,0) to[cisource,l=$g_m v_{gs}$] (3,-2);
                    \draw (3,0) -- (6,0);
                    \draw (2,-2) node[ground]{};
                    \draw (2,-2) node[anchor=south]{S};
                    \draw (1,0) node[anchor=south]{G};
                    \draw (5,0) node[anchor=south]{D};
                    \draw (0,0) node[anchor=east]{$v_{in}$};
                    \draw (6,0) node[anchor=west]{$v_{out}$};
                    \draw (5,0) to[R,l=$R_D$] (5,-2);
                    \draw (5,-2) node[ground]{};
                \end{circuitikz}
            \end{figure}

            \vspace{5mm}
            \[ \boxed{R_{in} = \infty} \]
            %
            \[ \boxed{R_{out} = R_D} \]
        \end{column}
    \end{columns}
\end{frame}

\begin{frame}
     \frametitle{Fuente Común con $\lambda \neq 0$}

     \centering
     \begin{figure}[H]
         \begin{circuitikz}
             \draw (0,0) -- (1,0);
             \draw (1,0) to[open,v=$v_{gs}$] (1,-2);
             \draw (1,-2) -- (5,-2);
             \draw (3,0) to[cisource,l=$g_m v_{gs}$] (3,-2);
             \draw (3,0) -- (8,0);
             \draw (2,-2) node[ground]{};
             \draw (2,-2) node[anchor=south]{S};
             \draw (1,0) node[anchor=south]{G};
             \draw (5,0) node[anchor=south]{D};
             \draw (0,0) node[anchor=east]{$v_{in}$};
             \draw (8,0) node[anchor=west]{$v_{out}$};
             \draw (7,0) to[R,l=$R_D$] (7,-2);
             \draw (7,-2) node[ground]{};
             \draw (5,0) to[R,l=$r_o$] (5,-2);
         \end{circuitikz}
     \end{figure}

     \vspace{5mm}
     \begin{columns}
        \begin{column}{0.5\textwidth}
            \[ \boxed{A_V = -g_m (R_D \parallel r_o)} \]
        \end{column}
        \begin{column}{0.5\textwidth}
            \[ \boxed{R_{in} = \infty} \]
            %
            \[ \boxed{R_{out} = R_D \parallel r_o} \]
        \end{column}
     \end{columns}
\end{frame}

\begin{frame}[t]
    \frametitle{Fuente Común: Ejemplo}

    Considere el siguiente circuito amplificador de fuente común, donde se requiere establecer una corriente $I_D=1\ mA$.

    \vspace{3mm}
    \begin{columns}
        \begin{column}{0.5\textwidth}
            \centering
            \begin{figure}[H]
                \begin{circuitikz}
                    \draw (0,0) node[nmos](nmos1){M1};
                    \draw (-1.5,0) to[short] (nmos1.gate);
                    \draw (-1.5,0) node[anchor=east]{$v_{in}$};
                    \draw (0,2) to[R,l=$R_D$] (nmos1.drain);
                    \draw (0,0.5) to[short,-o] (1,0.5);
                    \draw (0,-1) node[ground]{};
                    \draw (0,-1) -- (nmos1.source);
                    \draw (1,0.5) node[anchor=west]{$v_{out}$};
                    \draw (0,2.5) -- (0,2);
                    \draw (-0.5,2.5) -- (0.5,2.5);
                    \draw (0,2.5) node[anchor=south]{$V_{DD}$};
                \end{circuitikz}
            \end{figure}
        \end{column}
        \begin{column}{0.5\textwidth}
            \[ V_{DD} = 1.8\ V \]
            %
            \[ R_D = 1\ k\Omega \]
            %
            \[ \mu_n C'_{ox} = 100\ \mu A/V^2 \]
            %
            \[ V_{TH} = 0.5\ V \]
            %
            \[ \lambda=0 \]
            %
            \[ W/L = 10/0.18 \]
        \end{column}
    \end{columns}

    \vspace{5mm}
    Calcule  $V_{GS}$, $V_{DS}$, $g_m$, $A_V$ y demuestre que M1 opera en saturación.

\end{frame}

\begin{frame}[t]
    \frametitle{Fuente Común: Solución}

    \begin{columns}
        \begin{column}{0.5\textwidth}
            \textbf{Transconductancia}

            \[ g_m = \sqrt{2 \mu_n C'_{ox} \dfrac{W}{L} I_D } \]
            %
            \[ g_m = \sqrt{2 (100\ \mu A/V^2) \dfrac{10}{0.18} (1\ mA) } \]
            %
            \[ \boxed{g_m = 0.00333\ S} \]

            \textbf{Ganancia}

            \[ A_V = -g_m R_D \]
            %
            \[ A_V = -(3.33\ mS)(1\ k\Omega) \]
            %
            \[ \boxed{A_V = 3.33\ V/V} \]
        \end{column}
        \begin{column}{0.5\textwidth}
            \textbf{Polarización}

            \[ I_D = \dfrac{1}{2} \mu_n C'_{ox} \dfrac{W}{L} (V_{GS}-V_{TH})^2 \]
            %
            \[ V_{GS} = V_{TH} + \sqrt{\dfrac{2 I_D}{\mu_n C'_{ox} W/L}} \]
            %
            \[ V_{GS} = 0.5\ V + \sqrt{\dfrac{2 (1\ mA)}{(100\ \mu A/V^2) (10/0.18)}} \]
            %
            \[ \boxed{V_{GS} = 1.1\ V} \]

            \[ V_{DS} = V_{DD} - I_D R_D \]
            %
            \[ V_{DS} = 1.8\ V - (1\ mA)(1\ k\Omega) \]
            %
            \[ \boxed{V_{DS} = 0.8\ V} \]
            
        \end{column}
    \end{columns}
\end{frame}

\begin{frame}[t]
    \frametitle{Fuente Común con Fuente de Corriente}

    Considere ahora lo que sucede si en lugar de $R_D$ se coloca una fuente de corriente CD:

    \vspace{3mm}
    \begin{columns}
        \begin{column}{0.5\textwidth}
            \centering
            \begin{figure}[H]
                \begin{circuitikz}
                    \draw (0,0) node[nmos](nmos1){M1};
                    \draw (-1.5,0) to[short,o-] (nmos1.gate);
                    \draw (-1.5,0) node[anchor=east]{$v_{in}$};
                    \draw (0,2) to[isource,l=$I_{DD}$] (nmos1.drain);
                    \draw (0,0.5) to[short,-o] (1,0.5);
                    \draw (0,-1) node[ground]{};
                    \draw (0,-1) -- (nmos1.source);
                    \draw (1,0.5) node[anchor=west]{$v_{out}$};
                    \draw (0,2.5) -- (0,2);
                    \draw (-0.5,2.5) -- (0.5,2.5);
                    \draw (0,2.5) node[anchor=south]{$V_{DD}$};
                \end{circuitikz}
            \end{figure}
        \end{column}
        \begin{column}{0.5\textwidth}
            Al reemplazar $R_D$ por una fuente de corriente, $R_D \rightarrow \infty$ y se obtiene ganancia máxima.

            \[ A_V = -g_{m1} R_D \rightarrow \infty \]

            \vspace{5mm}
            Si el transistor M1 presenta modulación de largo de canal:

            \[ A_V = -g_{m1} r_{o1} \]
        \end{column}
    \end{columns}

    \vspace{5mm}
    La ganancia de esta configuración es la máxima posible, y está limitada por la modulación de largo de canal.
\end{frame}

\begin{frame}[t]
    \frametitle{Fuente Común con Transistor como Fuente de Corriente}

    Las fuentes de corriente en silicio se implementan con transistores MOSFET:

    \vspace{5mm}
    \begin{columns}
        \begin{column}{0.5\textwidth}
            \centering
            \begin{figure}[H]
                \begin{circuitikz}
                    \draw (0,0) node[nmos](nmos1){M1};
                    \draw (-1.5,0) to[short,o-] (nmos1.gate);
                    \draw (-1.5,0) node[anchor=east]{$v_{in}$};
                    \draw (0,2) node[pmos](pmos1){M2};
                    \draw (nmos1.drain) -- (pmos1.drain);
                    \draw (0,1) to[short,-o] (1,1);
                    \draw (0,-1) node[ground]{};
                    \draw (0,-1) -- (nmos1.source);
                    \draw (1,1) node[anchor=west]{$v_{out}$};
                    \draw (0,3) -- (pmos1.source);
                    \draw (-0.5,3) -- (0.5,3);
                    \draw (0,3) node[anchor=south]{$V_{DD}$};
                    \draw (-1,2) to[short,o-] (pmos1.gate);
                    \draw (-1,2) node[anchor=east]{$v_{G2}$};
                \end{circuitikz}
            \end{figure}
        \end{column}
        \begin{column}{0.5\textwidth}
            La tensión $V_{G2}$ se debe aplicar para obtener $I_{D2} \propto V_{SG2}$.

            \vspace{5mm}
            Si ambos transistores presentan modulación de largo de canal, las impedancias $r_{o1}$ y $r_{o2}$ están en paralelo.

            \vspace{5mm}
            La ganancia máxima de este circuito es:

            \[ A_V = -g_{m1} (r_{o1} \parallel r_{o2}) \]

            \vspace{5mm}
            La impedancia de salida es:

            \[ R_{out} = r_{o1} \parallel r_{o2} \]
        \end{column}
    \end{columns}
\end{frame}

\begin{frame}[t]
    \frametitle{Fuente Común con Transistor como Diodo}

    Si la compuerta de M2 se conecta a $V_{DD}$ (conexión como diodo):

    \vspace{5mm}
    \begin{columns}
        \begin{column}{0.5\textwidth}
            \centering
            \begin{figure}[H]
                \begin{circuitikz}
                    \draw (0,0) node[nmos](nmos1){M1};
                    \draw (-1.5,0) to[short,o-] (nmos1.gate);
                    \draw (-1.5,2) -- (-1.5,3);
                    \draw (-1.5,0) node[anchor=east]{$v_{in}$};
                    \draw (0,2) node[pmos](pmos1){M2};
                    \draw (nmos1.drain) -- (pmos1.drain);
                    \draw (0,1) to[short,-o] (1,1);
                    \draw (0,-1) node[ground]{};
                    \draw (0,-1) -- (nmos1.source);
                    \draw (1,1) node[anchor=west]{$v_{out}$};
                    \draw (0,3) -- (pmos1.source);
                    \draw (-2,3) -- (0.5,3);
                    \draw (-0.75,3) node[anchor=south]{$V_{DD}$};
                    \draw (-1.5,2) to[short] (pmos1.gate);
                \end{circuitikz}
            \end{figure}
        \end{column}
        \begin{column}{0.5\textwidth}
            Si se considera $\lambda=0$,
            %
            \[ A_V = -g_{m1} R_D \]
            %
            \[ A_V = -g_{m1} \cdot \dfrac{1}{g_{m2}} \]
            %
            \[ A_V = \dfrac{-g_{m1}}{g_{m2}} = \dfrac{\sqrt{2I_D\mu_n C'_{ox}(W/L)_1}}{\sqrt{2I_D\mu_n C'_{ox}(W/L)_2}} \]
            %
            \[ \boxed{A_V = -\sqrt{\dfrac{(W/L)_1}{(W/L)_2}}} \]
        \end{column}
    \end{columns}

    \vspace{3mm}
    La ganancia depende únicamente de las dimensiones de los transistores, independiente de la tecnología.
\end{frame}

\begin{frame}[t]
    \frametitle{Fuente Común con Degeneración de Fuente}

    El circuito se modifica para incluir degeneración de fuente:

    \vspace{5mm}
    \begin{columns}
        \begin{column}{0.36\textwidth}
            \centering
            \begin{figure}[H]
                \begin{circuitikz}
                    \draw (0,0) node[nmos](nmos1){};
                    \draw (-1.5,0) to[short] (nmos1.gate);
                    \draw (-1.5,0) to[vsource,l=$v_{in}$] (-1.5,-2);
                    \draw (-1.5,-2) node[ground]{};
                    \draw (0,2) to[R,l=$R_C$] (nmos1.drain);
                    \draw (0,0.5) to[short,-o] (1,0.5);
                    \draw (0,-2) node[ground]{};
                    \draw (nmos1.source) to[R,l=$R_E$] (0,-2);
                    \draw (1,0.5) node[anchor=west]{$v_{out}$};
                    \draw (0,2.5) -- (0,2);
                    \draw (-0.5,2.5) -- (0.5,2.5);
                    \draw (0,2.5) node[anchor=south]{$V_{DD}$};
                \end{circuitikz}
            \end{figure}
        \end{column}
        \begin{column}{0.64\textwidth}
            \centering
            \begin{figure}[H]
                \begin{circuitikz}
                    \draw (0,0) -- (1,0);
                    \draw (1,0) to[open,v=$v_{GS}$] (1,-2);
                    \draw (1,-2) -- (3,-2);
                    \draw (3,0) to[cisource,l=$g_m v_{GS}$] (3,-2);
                    \draw (3,0) -- (6,0);
                    \draw (2,-2) to[R,l=$R_S$] (2,-4);
                    \draw (2,-4) node[ground]{};
                    \draw (2,-2) node[anchor=south]{S};
                    \draw (1,0) node[anchor=south]{G};
                    \draw (5,0) node[anchor=south]{D};
                    \draw (0,0) node[anchor=east]{$v_{in}$};
                    \draw (6,0) node[anchor=west]{$v_{out}$};
                    \draw (5,0) to[R,l=$R_D$] (5,-2);
                    \draw (5,-2) node[ground]{};
                \end{circuitikz}
            \end{figure}
        \end{column}
    \end{columns}
\end{frame}

\begin{frame}[t]
    \frametitle{Fuente Común con Degeneración de Fuente}

    \begin{columns}
        \begin{column}{0.5\textwidth}
            1. LVK por la compuerta:
            %
            \[ v_{in} = v_{GS} + g_m v_{GS} R_S \]
            %
            \[ v_{in} = v_{GS} (1 + g_m R_S) \]
            %
            \[ v_{GS} = \dfrac{v_{in}}{1+g_m R_S} \]

            2. LCK en el drenador:
            %
            \[ \dfrac{v_{out}}{R_D} + g_m v_{GS} = 0 \]
            %
            \[ v_{out} = -g_m v_{GS} R_D \]
        \end{column}
        \begin{column}{0.5\textwidth}
            Combinando las dos expresiones:
            %
            \[ v_{out} = -g_m \left( \dfrac{v_{in}}{1+g_m R_S} \right) R_D \]
            %
            \[ \dfrac{v_{out}}{v_{in}} = \dfrac{-g_m R_D}{1 + g_m R_S} \]
            %
            \[ \boxed{A_V = \dfrac{-R_D}{\dfrac{1}{g_m} + R_S}} \]
        \end{column}
    \end{columns}

    \vspace{5mm}
    \begin{itemize}
        \item La ecuación de ganancia es la misma que para E.C. con $R_E$.
        \item La impedancia de salida es la resistencia (misma $R_{out}$ que E.C.)
        \item La impedancia de entrada es infinita (mejor $R_{in}$ que E.C.)
    \end{itemize}
\end{frame}

\begin{frame}[t]
    \frametitle{Fuente Común con Transistor de Degeneración}

    Considere el siguiente circuito (suponiendo $\lambda=0$):

    \vspace{5mm}
    \begin{columns}
        \begin{column}{0.5\textwidth}
            \centering
            \begin{figure}[H]
                \begin{circuitikz}
                    \draw (0,0) node[nmos](nmos1){M1};
                    \draw (-1.5,0) to[short,o-] (nmos1.gate);
                    \draw (-1.5,0) node[anchor=east]{$v_{in}$};
                    \draw (0,2) to[R,l=$R_D$] (nmos1.drain);
                    \draw (0,0.5) to[short,-o] (1,0.5);
                    \draw (0,-2) node[ground]{};
                    \draw (0,-1.5) node[pmos](pmos1){M2};
                    \draw (pmos1.gate) -- (-1,-1.5);
                    \draw (-1,-1.5) -- (-1,-2);
                    \draw (-1,-2) node[ground]{};
                    \draw (nmos1.source) -- (pmos1.source);
                    \draw (pmos1.drain) -- (0,-2);
                    \draw (1,0.5) node[anchor=west]{$v_{out}$};
                    \draw (0,2.5) -- (0,2);
                    \draw (-0.5,2.5) -- (0.5,2.5);
                    \draw (0,2.5) node[anchor=south]{$V_{DD}$};
                \end{circuitikz}
            \end{figure}
        \end{column}
        \begin{column}{0.5\textwidth}
            M2 es un PMOS conectado como diodo:

            \vspace{5mm}
            En peque\~{n}a se\~{n}al se puede reemplazar este transistor por $1/g_{m2}$.

            \vspace{5mm}
            La ganancia del circuito es entonces:

            \[ \boxed{A_V = \dfrac{-R_D}{\dfrac{1}{g_{m1}} + \dfrac{1}{g_{m2}}}} \]
            %
            \[ R_{in} = \infty \]
            %
            \[ R_{out} = R_D \]
        \end{column}
    \end{columns}
\end{frame}

\begin{frame}[t]
    \frametitle{Fuente Común con RS (efecto de $\lambda$ en $R_{out}$)}
    
    Al incluir modulación de largo de canal, $R_{out}$ cambia:

    \centering
    \begin{figure}[H]
        \begin{circuitikz}
            \draw (-1,0) -- (1,0);
            \draw (1,0) to[open,v=$v_{GS}$] (1,-2);
            \draw (1,-2) -- (5,-2);
            \draw (3,0) to[cisource,l=$g_m v_{GS}$] (3,-2);
            \draw (3,0) -- (7,0);
            \draw (2,-2) to[R,l=$R_S$] (2,-4);
            \draw (2,-4) node[ground]{};
            \draw (2,-2) node[anchor=south]{S};
            \draw (1,0) node[anchor=south]{G};
            \draw (5,0) node[anchor=south]{D};
            \draw (-1,0) node[ground]{};
            \draw (5,0) to[R,l=$r_o$] (5,-2);
            \draw (7,-2) node[ground]{};
            \draw (7,0) to[vsource,l=$v_x$,i=$i_x$] (7,-2);
        \end{circuitikz}
    \end{figure}

    \vspace{3mm}
    \flushleft
    Se procede a calcular la impedancia desde el drenador: $R_{out}=v_x/i_x$.

\end{frame}

\begin{frame}[t]
    \frametitle{Fuente Común con RS (efecto de $\lambda$ en $R_{out}$)}

    \begin{columns}
        \begin{column}{0.5\textwidth}
            1. LVK por la compuerta:
            %
            \[ v_{GS} + v_{RS} = 0 \]
            %
            \[ v_{GS} = -v_{RS} \]
            %
            \[ v_{GS} = -i_x \cdot R_S \]

            2. La corriente por $r_o$:
            %
            \[ i_{ro} = i_x - g_m v_{GS} \]
            %
            \[ i_{ro} = i_x - g_m (-i_x \cdot R_S) \]
            %
            \[ i_{ro} = i_x (1 + g_m R_S) \]
        \end{column}
        \begin{column}{0.5\textwidth}
            3. LVK por el drenador:
            %
            \[ v_x = v_{ro} + v_{RS} \]
            %
            \[ v_x = i_{ro} r_o + i_x R_S \]
            %
            \[ v_x = i_x (1 + g_m R_S) r_o + i_x R_S \]
            %
            \[ \boxed{\dfrac{v_x}{i_x} = (1 + g_m R_S) r_o + R_S} \]

            Distribuyendo $r_o$:
            %
            \[ R_{out} = r_o + g_m R_S r_o + R_S \]
            %
            \[ \boxed{R_{out} = R_S (1 + g_m r_o) + r_o} \]
        \end{column}
    \end{columns}

    \centering
    \vspace{7mm}
    Una expresión aproximada es $ \boxed{R_{out} \approx g_m r_o R_S + r_o } $

\end{frame}

\begin{frame}[t]
    \frametitle{Fuente Común: Polarización por Divisor de Tensión}

    \begin{columns}
        \begin{column}{0.5\textwidth}
            \centering
            \textbf{Por divisor de tensión}

            \begin{figure}[H]
                \begin{circuitikz}
                    \draw (0,0) node[nmos](nmos1){};
                    \draw (0,2) to[R,l=$R_D$] (nmos1.drain);
                    \draw (nmos1.source) to[R,l=$R_S$] (0,-2);
                    \draw (0,-2) node[ground]{};
                    \draw (-2,2) to[R,l=$R_1$] (-2,0);
                    \draw (-2,0) to[R,l=$R_2$] (-2,-2);
                    \draw (-2,-2) node[ground]{};
                    \draw (-2,2) -- (-2,2.5);
                    \draw (0,2) -- (0,2.5);
                    \draw (-2.5,2.5) -- (0.5,2.5);
                    \draw (-1,2.5) node[anchor=south]{$V_{DD}$};
                    \draw (-2,0) -- (nmos1.gate);
                    \draw (-3,0) to[short,o-] (-2,0);
                    \draw (-3,0) node[anchor=east]{$v_{in}$};
                    \draw (0,0.5) to[short,-o] (1,0.5);
                    \draw (1,0.5) node[anchor=west]{$v_{out}$};
                \end{circuitikz}
            \end{figure}
        \end{column}
        \begin{column}{0.5\textwidth}
            La impedancia de entrada ya no es infinita

            \[ R_{in} = R_1 \parallel R_2 \]

            La ganancia se mantiene

            \[ A_V = \dfrac{-R_D}{\dfrac{1}{g_m} + R_S} \]

            \vspace{5mm}
            Si la fuente tuviera una resistencia en serie $R_G$

            \[ A_V = \dfrac{R_1 \parallel R_2}{R_G + R_1 \parallel R_2} \cdot \dfrac{-R_D}{\dfrac{1}{g_m} + R_S} \]
        \end{column}
    \end{columns}
\end{frame}


\begin{frame}[t]
    \frametitle{Fuente Común: Configuración Completa}

    \centering
    \begin{figure}[H]
        \begin{circuitikz}
            \draw (0,0) node[nmos](nmos1){};
            \draw (0,2) to[R,l=$R_D$] (nmos1.drain);
            \draw (nmos1.source) to[R,l=$R_S$] (0,-2);
            \draw (1.33,-0.5) to[C,l=$C_b$] (1.33,-2);
            \draw (0,-0.5) -- (1.33,-0.5);
            \draw (1.33,-2) node[ground]{};
            \draw (0,-2) node[ground]{};
            \draw (-2,2) to[R,l=$R_1$] (-2,0);
            \draw (-2,0) to[R,l=$R_2$] (-2,-2);
            \draw (-2,-2) node[ground]{};
            \draw (-2,2) -- (-2,2.5);
            \draw (0,2) -- (0,2.5);
            \draw (-2.5,2.5) -- (0.5,2.5);
            \draw (-1,2.5) node[anchor=south]{$V_{DD}$};
            \draw (-2,0) -- (nmos1.gate);
            \draw (-4,0) to[C] (-2,0);
            \draw (0,0.5) to[C] (2,0.5);
            \draw (2,0.5) to[short,-o] (4,0.5);
            \draw (4,0.5) node[anchor=west]{$v_{out}$};
            \draw (3,0.5) to[R,l=$R_L$] (3,-2);
            \draw (3,-2) node[ground]{};
            \draw (-6,0) to[R,l=$R_G$] (-4,0);
            \draw (-6,0) to[vsource,l=$v_{in}$] (-6,-2);
            \draw (-6,-2) node[ground]{};
        \end{circuitikz}
    \end{figure}

    \[ A_V = \dfrac{R_1 \parallel R_2}{R_G + R_1 \parallel R_2} \cdot -g_m R_D \]
\end{frame}


\begin{frame}[t]
    \frametitle{Fuente Común: Ejemplo 7.11 página 324}
    % Ejemplo 7.11 pag 324

    Dise\~{n}e una etapa C.S. con degeneración de fuente, polarizada por divisor de tensión, con condensador de derivación, que tenga $A_V=5$ con $R_{in}=50\ k\Omega$.

    \vspace{3mm}
    Los parámetros de la tecnología son $K'=\mu_n C'_{ox}=100\ \mu A/V^2$, $V_{TH}=0.5\ V$, $\lambda=0$, $V_{DD}=1.8\ V$ y $V_{RS}=400\ mV$.
    
    \vspace{3mm}
    Para este problema se tiene un presupuesto de potencia de 5 mW, es decir, $I_{max}=P/V=5\ mW/1.8\ V=2.78\ mA$. Puede asumir que $I_D=2.7\ mA$ y que la corriente de polarización del divissor de tensión es $I_{R1}=I_{R2}=80\ \mu A$. 
    
    

    \vspace{10mm}
    \centering
    \textbf{Respuestas}

    \begin{table}[H]
        \begin{tabular}{cc}
            $R_S = 148\ \Omega$ & $g_m = 10.8\ mS$ \\
            $R_D = 463\ \Omega$ & $W/L=216$ \\
            $R_1 = 64.3\ k\Omega$ & $V_G = 1.4\ V$ \\
            $R_2 = 225\ k\Omega$ & $V_S = 0.4\ V$ \\
        \end{tabular}
    \end{table}
    
\end{frame}
